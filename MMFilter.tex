\chapter{Multi-Modell-Filter}
\section{Static Multiple Model (SMM)}
Es werden $r$ separate Einzelfilter berechnet, zwischen den Filtern findet keine Interaktion statt.

\section{Generalized Pseudo Bayesian (GPB1)}
Es werden $r$ separate Einzelfilter berechnet die nach jedem Zeitschritt kombiniert werden, die kombinierte Schätzung ist die a-priori Information für den nächsten Zeitschritt.

Die Kombination erfolgt gewichtet:
\begin{equation*}
    p(x(k) | Z^k) = \sum_{j=1}^r p(x(k)|M_j(k), Z^k) P\{M_j(k) | Z^k\}
\end{equation*}
im folgenden wird die Moduswahrscheinlichkeit abgekürzt:
\begin{equation*}
    \mu_j (k) = P\{M_j(k), Z^k\}
\end{equation*}
Die $\mu_i$ berechnen sich dann auf Basis der Likelihood-Funktion und Übergangswahrscheinlickeiten $p_{ij}$:
\begin{equation*}
    \mu_j(k) = \frac{1}{c} \Lambda_j(k) \sum_{i=1}^r p_{ij} \mu_i(k-1)
\end{equation*}
Wobei $c$ so gewählt wird, dass $\sum \mu_j(k) = 1$.

Die Übergangswahrscheinlichkeiten, welche a-priori berechnet werden, modellieren das Filter als Markov-Kette. 

Die finale Schätzung von Zustand und Kovarianz ist dann gegeben durch:
\begin{eqnarray*}
    \hat{x}(k|k) &=& \sum_{j=1}^r \hat{x}^j (k|k) \mu_j(k) \\
    P(k|k) &=& \sum_{j=1}^r \mu_j(k) \left( P^j(k|k) + \left(\hat{x}^j(k|k) - \hat{x}(k|k)\right) \T{\left(\hat{x}^j(k|k) - \hat{x}(k|k)\right)} \right)
\end{eqnarray*}

Wobei der hintere Term die Unsicherheit durch die Modellauswahl berücksichtigt.

\section{Interacting Multiple Model Filter (IMM)}
Beim IMM werden nicht alle Filter auf Basis von einem a-priori Schätzwert berechnet, sondern jedes Filter erhält einen eigenen a-priori Schätzwert.

Wie beim GPB1 wird wieder ein Gewicht für jedes Modell berechnet:
\begin{equation*}
    \mu_i (k|k-1) = \sum_{j=1}^r p_{ji} \mu_j(k-1|k-1)
\end{equation*}
sowie Übergangsgewichte:
\begin{equation*}
    \mu_{j|i}(k-1) = \frac{p_{ij} \mu_i(k-1|k-1)}{\mu_j(k|k-1)}
\end{equation*}
somit kann die a-priori Schätzung für jedes Filter berechnet werden:
\begin{equation*}
    \hat{x}^{0i} (k-1|k-1) = \sum_{j=1}^r \hat{x}^j (k-1|k-1) \mu_{j|i}(k-1)
\end{equation*}
sowie die zugehörige Kovarianz:
\begin{eqnarray*}
    P^{0i}&&(k-1|k-1) = \\
    &&\sum_{j=1}^r \left(P^j(k-1|k-1) + 
    \left(\hat{x}^{0i}(k-1|k-1) x^j(k-1|k-1)\right)
    \T{\left(\hat{x}^{0i}(k-1|k-1) x^j(k-1|k-1)\right)} \right) \\
    &&\mu_{j|i}(k)
\end{eqnarray*}

Die Kombination für die Ausgangsschätzung erfolgt dann gemäß:
\begin{eqnarray*}
    \hat{x}(k+1|k+1) &=& \sum_{i=1}^r \hat{x}^i(k+1|k+1)\mu_i(k+1) \\
    P(k+1|k+1) &=& \sum_{i=1}^r \mu_i(k+1) \\
        && (P^i(k+1|k+1) + \left(\hat{x}^i(k+1|k+1) - \hat{x}(k+1|k+1)\right) \\
        && \T{\left(\hat{x}^i(k+1|k+1) - \hat{x}(k+1|k+1)\right)} )
\end{eqnarray*}
