\chapter{Grundlagen der Parameterschätzung}
\section{Definitionen}
\begin{itemize}
    \item $x$: Parameter (fester, zeitinvarianter Wert), der geschätzt werden soll (Skalar oder Vektor).
    \item $z$: Messung des Parameters $x$ mit einer einer Messfunktion $h(\cdot)$ (Skalar oder Vektor).
        Die Messungen sind in der Regel gestört, z.B. durch ein Rauschen $w$: $z = h(x, w)$.
    \item $Z^k$: $Z^k = \{z_1, \ldots, z_k\}$ ist die Gesamtheit aller $k$-Messungen.
    \item $Z$: Gesamtheit aller erfolgten Messungen (Anzahl nicht spezifiziert).
\end{itemize}

\paragraph{Schätzer:}
\begin{equation*}
    \hat{x}(k) = \hat{x}[k, Z^k]\ Z^k = \{z_1, \ldots, z_k\}
\end{equation*}

Das Ergebnis der Schätzung ist der Ausgangswert $\hat{x}(k)$ des Schätzers. 
Die Begriffe Schätzung und Schätzergebnis werden synonym verwendet.

\paragraph{Schätzfehler:}
Der Schätzfehler ist die Abweichung zwischen dem (unbekannten) wahren Wert und dem Schätzergebnis:
\begin{equation*}
    \tilde{x} = x - \hat{x}
\end{equation*}

\section{Modelle zur Parameterschätzung}
\subsection{Likelihood-Funktion (Non-Bayes Methode)}
Der zu schätzende Parameter wird als deterministisch angenommen, d.h. es existiert ein exakter, aber unbekannter, 
Wert $x_0$.

Die Likelihood Funktion gibt die Likelihood an mit der eine Messung $Z$ zu einem Zustand $x$ gehört:
\begin{equation*}
    \Lambda_Z(x) = p(Z|x)
\end{equation*}

\subsubsection{ML-Schätzer}
Das Schätzergebnis ergibt sich als das $x$, mit höchstem Likelihood:
\begin{equation*}
    \hat{x}^\text{ML} (Z) = \argmax_x (\Lambda_Z(x))
\end{equation*}

Für gaussverteilte Messgrößen mit gleicher Varianz bildet der ML-Schätzer den Durchschnitt.

\subsection{Regel von Bayes}
Der zu schätzende Parameter wird als stochastisch mit a priori PDF $p(x)$ angenommen.

Dann folgt mit der Regel von Bayes:
\begin{equation*}
    p(x|Z) = \frac{p(Z|x) \cdot p(x)}{p(Z)} = \frac{1}{c} p(Z|x) p(x)
\end{equation*}

Mit:
\begin{itemize}
    \item $p(Z|x)$: PDF der Messung, bedingt durch den wahren Parameter $x$.
    \item $p(x|Z)$: A posteriori PDF der Schätzung
    \item $p(x)$: A priori PDF der Schätzung
    \item $p(Z)$: PDF der Messungen, dient der Normierung
    \item $\frac{1}{c}$: Normierungsfaktor
\end{itemize}

Vorsicht: $c = p(Z)$ gilt nur für statistisch unabhängigen Messungen.

\subsubsection{MAP-Schätzer}
Für den Maximum-A-Posteriori PDF wird das wahrscheinlichste $x$ zu den Messungen genommen:
\begin{equation*}
    \hat{x}^\text{MAP}(Z) = \argmax_x p(x|Z) = \argmax_x \left(p(Z|x) p(x)\right)
\end{equation*}

Für gaussverteilte Messgrößen bildet der MAP-Schätzer das gewichtete Mittel aus Prior und Messung, die Wichtung ist hierbei durch die Varianz der jeweils anderen Größe normiert durch die Gesamtvarianz gegeben.

\subsection{Least-Squares Schätzer}
Es wird der Zustand $x$, der den kleinsten Quadratischen Messfehler erzeugt gesucht ($z(j) = h(j, x) + w(j)$):
\begin{equation*}
    \hat{x}^\text{LS} = \argmin_x \left( \sum_{j=1}^k {\left(z(j) - h(j,x)\right)}^2 \right)
\end{equation*}

Für normalverteilte, mittelwertfreie Messfehler sind LS und ML identisch.

\subsection{Minimum Mean Square Error}
Der MMSE Schätzer versucht den quadratischen Fehler von $\hat{x}$ zu $x$ zu minimieren, nimmt hierbei allerdings $x$ als stochastisch an:
\begin{equation*}
    \hat{x}^\text{MMSE}(Z) = \argmin_{\hat{x}} \E{{(\hat{x} - x)}^2 | Z}
\end{equation*}

Für normalverteilte größen stimmen MMSE und MAP überein.

\section{Erwartungstreue von Schätzern}
Ein Schätzer heißt erwartungstreu, wenn für den Schätzfehler $\tilde{x}$ gilt:
\begin{equation*}
    \E{\tilde{x}} = 0
\end{equation*}

Dabei lässt sich unterscheiden zwischen:
\begin{itemize}
    \item (Generelle) Erwartungstreue:
        \begin{equation*}
            \E{\tilde{x}} = 0\ \forall k
        \end{equation*}
    \item Asymptotische Erwartungstreue:
        \begin{equation*}
            \E{\tilde{x}} = 0\ \text{für } k \to \infty
        \end{equation*}
\end{itemize}

\section{Konsistenz}
Wenn der Schätzer erwartungstreu ist, kann die Konsistenz als:
\begin{equation*}
    \lim_{k \to \infty} \tilde{x}(k, Z^k) = 0
\end{equation*}

Es lässt sich zeigen, dass die Varianz bzw. der MSE eines Schätzers nie kleiner als die Cramer-Rao Grenze werden kann:
\begin{equation*}
    \E{(\hat{x}(Z) - x) \T{(\hat{x}(Z) - x)}} \geq J^{-1}
\end{equation*}
mit der Fisher-Informationsmatrix $J$ als:
\begin{equation*}
    J = -\left. \E{\nabla_x \T{\nabla}_x \ln(\Lambda(x))} \right|_{x=x_0} 
        = \left. \E{\left(\nabla_x \ln(\Lambda(x))\right) \T{(\nabla_x \ln(\Lambda(x)))}} \right|_{x=x_0}
\end{equation*}
Wobei die Größenrelation über die positive semidefinitheit der differenz der beiden Matrizen definiert ist:
\begin{equation*}
    A \geq B \Leftrightarrow A - B \geq 0
\end{equation*}
