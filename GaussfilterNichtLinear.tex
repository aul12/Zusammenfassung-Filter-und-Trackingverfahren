\chapter{Gaußfilter für nichtlineare Systeme}
Bei vielen realen Systemen sind System und/oder Messgleichung nichtlinear, so bleiben die PDFs keine Normalverteilung. Oftmals lässt sich aber die PDF trotzdem durch eine Gaußverteilung
approximieren.

\section{Extended Kalman Filter (EKF)}
Das EKF löst das Problem der Nichtlinearitäten durch linearisierung.

Allgemein lautet das Zustandsraummodell:
\begin{eqnarray*}
    x(k+1) &=& f(k, x(k), u(k), v(k)) \\
    z(k+1) &=& h(k, x(k), w(k))
\end{eqnarray*}

Beim EKF wird additives Rauschen angenommen (und hier auch keine Steuergröße):
\begin{eqnarray*}
    x(k+1) &=& f(k, x(k)) + v(k) \\
    z(k) &=& h(k, x(k)) + w(k)
\end{eqnarray*}

Durch linearisierung (Taylor) erhält man daraus ein linearisiertes Modell:
\begin{equation*}
    x(k+1) = f(k, \hat{x}(k|k)) + f_x(k) (x(k) - \hat{x}(k|k))
\end{equation*}
wobei $f_x$ die Ableitung von $f$ nach $x$ ist:
\begin{equation*}
    f_x(k) = \left. \T{\left( \nabla_x \T{f(k|x)} \right)} \right|_{x=\hat{x}(k|k)}
        = \left. \frac{\partial f}{\partial x} \right|_{x=\hat{x}(k|k)}
\end{equation*}

\subsection{Prädiktionsschritt}
\begin{eqnarray*}
    \hat{x}(k+1|k) &=& f(k, \hat{x}(k|k)) \\
    P(k+1|k) &=& f_x(k) P(k|k) \T{f_x(k)} + Q(k) \\
    \hat{z}(k+1|k) &=& h(k+1, \hat{x}(k+1|k)) \\
    S(k+1) &=& h_x(k+1) P(k+1|k) \T{h_x(k+1)} + R(k+1)
\end{eqnarray*}

\subsection{Innovationsschritt}
\begin{eqnarray*}
    K(k+1) &=& P(k+1|k) \T{h_x(k+1)} {S(k+1)}^{-1} \\
    \hat{x}(k+1|k+1) &=& \hat{x}(k+1|k) + K(k+1) (z(k+1) - \hat{z}(k+1|k)) \\
    P(k+1|k+1) &=& P(k+1|k) - K(k+1)S(k+1)\T{K(k+1)}
\end{eqnarray*}

Selbstverständlich muss auch hier eine Anfangsschätzung $x(0|0)$ und $P(0|0)$ vorliegen.

Die Güte es EKF hängt von zwei Faktoren ab:
\begin{itemize}
    \item Die Unsicherheit der Schätzung (größere Unsicherheit bedeutet es wird eine größere Spanne an Eingangswerten
        abgedeckt)
    \item Die Größe der lokalen Nichtlinearität
\end{itemize}

\section{Unscented Kalmanfilter (UKF)}
Statt einer Taylorreihenentwicklung wird eine \glqq stochastische\grqq linearisierung durchgeführt, d.h. anstatt das System
zu linearisieren werden mehrere Punkte (auf der $\sigma$-Ellipse), transformiert und danach daraus eine 
Normalverteilung rekonstruiert.

\begin{enumerate}
    \item Wahl von $2n+1$ $\sigma$-Punkte $\chi_i(k)$ und Gewichten $W_i$ mit $\sum W_i = 1$.
    \item Transformation aller $\chi_i(k)$ durch Systemgleichung:
        \begin{equation*}
            \chi_i(k+1) = f(\chi_i(k))
        \end{equation*}
    \item Berechnung des neuen Erwartungswertes:
        \begin{equation*}
            \hat{x}(k+1|k) = \sum_{i=0}^{2n} W_i \chi_i(k+1)
        \end{equation*}
    \item Berechnung der neuen Kovarianz:
        \begin{equation*}
            P(k+1|k) = \sum_{i=0}^{2n} W_i \left(\chi_i(k+1) - \hat{x}(k+1|k) \right) 
                \T{\left(\chi_i(k+1) - \hat{x}(k+1|k) \right)}
        \end{equation*}
\end{enumerate}
Dieses Verfahren kann sowohl für die System, als auch für die Messgleichung angewandt werden.

\subsection{Wahl der $\chi_i$}
Eine mögliche Option für die Wahl der $\chi_i$ ist ($\kappa$ ist ein Design Parameter):
\begin{eqnarray*}
    \chi_0(k) &=& \hat{x}(k|k) \\
    \chi_{i}(k) &=& \hat{x}(k|k) + {\left( \sqrt{(n+\kappa) P(k|k)} \right)}_i \\
    \chi_{i+n}(k) &=& \hat{x}(k|k) - {\left( \sqrt{(n+\kappa) P(k|k)} \right)}_i \\
    W_0 &=& \frac{\kappa}{n + \kappa} \\
    W_i &=& W_{i+n} = \frac{1}{2 (n + \kappa)} 
\end{eqnarray*}

Die Wurzel von $P$, also eine Matrix $A$ mit $P=A\T{A}$ bzw. $P=\T{A}A$, kann z.B. mit einer Cholsky Zerlegung gefunden
werden.

Der Designparameter $\kappa$ kann z.B. als:
\begin{equation*}
    \kappa = 3 - n
\end{equation*}
gewählt werden (Heuristik).

\section{Nichtlineare Fahrzeugmodelle}
\subsection{Konstante Geschwindigkeit und Konstante Gierrate}
\begin{eqnarray*}
    x &=&
        \begin{pmatrix}
            x \\ y \\ v \\ \psi \\ \dot{\psi}
        \end{pmatrix} \\
    x(k+1) &=& f(k, x(k)) = x(k) + \Delta x(k+1) + v(k+1) \\
    \text{mit: } \Delta x(k+1) &=& 
        \begin{pmatrix}
            \frac{v(k)}{\dot{\psi}(k)} \left(\sin(\psi(k) + \dot{\psi}(k) T) - \sin(\psi(k))\right) \\
            \frac{v(k)}{\dot{\psi}(k)} \left(\cos(\psi(k) - \cos(\psi(k) + \dot{\psi}(k) T) \right) \\
            0 \\
            \dot{\psi}(k) T \\
            0
        \end{pmatrix}
\end{eqnarray*}
Für $\abs{\dot{\psi}} < \varepsilon$ muss eine Fallunterscheidung vorgenommen werden:
\begin{equation*}
    \Delta x(k+1) = 
        \begin{pmatrix}
            v(k) T \cos(\psi(k)) \\
            v(k) T \sin(\psi(k)) \\
            0 \\
            \dot{\psi}{k} T \\
            0
        \end{pmatrix}
\end{equation*}
Für die Rauschgewinne von Translation und Rotation ergibt sich (unter der Annahme $\dot{\psi}(k)$ klein):
\begin{eqnarray*}
    \Gamma_a(k) &=&
        \begin{pmatrix}
            \frac{1}{2} T^2 \cos(\psi(k)) \\
            \frac{1}{2} T^2 \sin(\psi(k)) \\
            T
        \end{pmatrix} \\
    \Gamma_{\ddot{\psi}} &=& 
        \begin{pmatrix}
            \frac{1}{2} T^2 \\
            T
        \end{pmatrix}
\end{eqnarray*}

Eine Erweiterung zu konstante Gierrate oder konstanter Beschleunigung ist möglich.
