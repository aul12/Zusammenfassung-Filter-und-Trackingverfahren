\chapter{Parameterfreie Filter, Partikelfilter}
Die PDFs werden durch ein Set von Partikeln (möglichen Zustände) angenähert:
\begin{equation*}
    \kappa(K) = \{x^1(k), x^2(k), x^N(k)\}
\end{equation*}
idealerweise gilt:
\begin{equation*}
    p(x(k)|z(k)) \propto \kappa(k)
\end{equation*}

\section{Monte-Carlo Integration}
Sei $g: \mathbb{R}^n \to \mathbb{R}$ eine integrierbare Funktion, d.h.:
\begin{equation*}
    I = \int g(x) \text{d}x
\end{equation*}
existiert, und es gibt zwei Funktionen $f: \mathbb{R}^n \to \mathbb{R}$, sowie
$\pi(x): \mathbb{R}^n \to \mathbb{R}$, wobei $\pi(x)$ eine PDF ist, also gilt:
\begin{equation*}
    \pi(x) \geq 0\ \forall x \land \int \pi(x) \text{d}x = 1
\end{equation*}
mit
\begin{equation*}
    g(x) = f(x) \pi(x)
\end{equation*}
dann kann $I$ durch Monte-Carlo Integration angenähert werden.
Wähle hierfür $N$ mit $N \gg 1$, und ziehe $N$ Samples $\{x^1, \ldots, x^N\}$, 
die der Verteilung $\pi(x)$ folgen.

Dann gilt:
\begin{equation*}
    I = \int g(x) \text{d}x = \int f(x) \pi(x) \text{d}x \approx
        I_N = \frac{1}{N} \sum_{i=1}^N f(x^i)
\end{equation*}

$I_N$ ist ein offsetfreier, erwartungswerttreuer Schätzwert für $I$.

\section{Importance Sampling}
Falls Samples nur anhand einer einfachereren Dichtefunktion $q(x)$ gezogen werden,
dann kann die Monte-Carlo Integration durchgeführt werden, falls gilt:
\begin{equation*}
    \pi(x) > 0 \Rightarrow q(x) > 0\ \forall x \in \mathbb{R}^n
\end{equation*}
Das Integral lässt sich dann formulieren als:
\begin{equation*}
    I = \int f(x) \pi(x) \text{d}x = \int \left( f(x) \frac{\pi(x)}{q(x)} \right) q(x) \text{d}x
\end{equation*}
dafür muss gelten $\frac{\pi(x)}{q(x)}$ ist beschränkt. Dieser Faktor lässt sich jetzt als Gewicht definieren:
\begin{equation*}
    \tilde{w}(x^i) = \frac{\pi(x^i)}{q(x^i)}
\end{equation*}
Falls $\pi(x)$ nicht bekannt ist, dann lassen sich normalisierte Gewichte als
\begin{equation*}
    w(x^i) = \frac{\tilde{w}(x^i)}{\sum_{j=1}^N \tilde{w}(x^j)}
\end{equation*}
definieren. Damit erhält man weiterhin
\begin{equation*}
    I_N = \frac{1}{N} \sum_{i=1}^N f(x^i) w(x^i)
\end{equation*}

\section{Sequentielles Importance Sampling}
Die \glqq{}abgetastete\grqq{} a-posteriori PDF ist dann gegeben durch:
\begin{equation*}
    p(x(k)|z(k)) \approx \sum_{i=1}^N w^i (k) \delta(x(k) - x^i(k))
        \text{ mit } \sum_{i=1}^N w^i(k) = 1
\end{equation*}

Die Importance Funktion kann über mehrere Zeitschritte mit prädiziert werden:
\begin{equation*}
    q(x(k)|z(k)) =  q(x(k)|x(k-1),z(k)) q(x(k-1)|z(k-1))
\end{equation*}
Damit folgt (ohne Normierung):
\begin{equation*}
    w^i(k) \propto w^i(k-1) \frac{p(z(k) | x^i(k)) p(x^i(k) | x^i(k-1))}{q(x^i(k) | x^i(k-1), z(k))}
\end{equation*}

\section{Resampling}
\subsection{Degeneration}
In der Praxis ergibt sich das Problem der Degeneration: die Gewichte aller, bis auf ein Partikel werden verschwindend klein, die PDF kann nicht mehr ausreichend gut dargestellt werden.

Die effektive Anzahl an Partikeln (ein Maß für die nicht-degeneration) berechnet sich als:
\begin{equation*}
    N_\text{eff} = \frac{1}{\sum_{j=1}^N {(w^j(k))}^2}
\end{equation*}

\subsection{Vorgehen}
Beim resampling soll ein Mapping vom alten (gewichteten-) Partikelset $\left(x^i(k), w^i(k)\right)$ zu einem neuen Partikelset $\left(x^{i*}(k), \frac{1}{N}\right)$ erfolgen.

Dafür werden $N$ neue Zufallszahlen $r_i$ im Interval $[0,1]$ bestimmt. Für jede Zufallszahl wird dann das neue Partikel $x^i$ gewählt durch:
\begin{eqnarray*}
    \tilde{i} &=& \argmin_n \left(\sum_{i=1}^n w^i \geq r_i \right) \\
    x^{i*} &=& x^{\tilde{i}}
\end{eqnarray*}

\section{Wahl der Importance-Funktion}
Die optimale Importance-Funktion lautet:
\begin{equation*}
    q(x(k)|x^i(k-1), z(k)) = p(x(k)|x^i(k-1),z(k))
\end{equation*}
die Bestimmung dieser Funktion ist im allgemeine aber nicht möglich.

Daher wird oftmals die suboptimale Importance Funktion:
\begin{equation*}
    q(x(k)|x^i(k-1), z(k)) = p(x(k)|x^i(k-1))
\end{equation*}

\section{Implementation}
Folgende, schwache, Voraussetzungen müssen erfüllt sein:
\begin{itemize}
    \item Systemgleichung $f(\cdot)$ und Messgleichung $h(\cdot)$ müssen analytisch
        gegeben sein.
    \item Es muss möglich sein Samples aus $v(k)$ zu ziehen.
    \item Die Dichte $p(z(k)|x^i(k))$ muss Punktweise auswertbar sein.
\end{itemize}

\subsection{Prädiktion}
\begin{equation*}
    x^i(k|k-1) = f(x^i(k-1)) + v^i(k-1)
\end{equation*}

\subsection{Innovation}
\begin{equation*}
    w^i(k) = \eta p(z(k) - h(x^i(k))) w^i(k-1)
\end{equation*}
mit $\eta$, so dass $\sum w_i(k) = 1$.

\section{Eigenschaften}
\subsection{Schätzung der Kontinuierlichen PDF}
In vielen Anwendungen ist eine kontinuierliche PDF gewünscht, diese muss im Nachhinein aus den Partikeln bestimmt werden, z.B. durch:
\begin{itemize}
    \item Gaussapproximation
    \item GMM
    \item Histogrammapproximation
    \item Kernel-Dichte Approximation (Parzen)
\end{itemize}

\subsection{Varianz durch den Sampling-Prozess}
Die Anzahl der Partikel bestimmt Maßgeblich wie gut eine PDF angenähert werden kann.
