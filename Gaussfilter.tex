\chapter{Gaußfilter für lineare Systeme}
Das allgemeine Bayes Filter lässt sich für allgemeine PDFs nur schwer auswerten. 
Daher werden die PDFs als Normalverteilungen angenommen:
\begin{equation*}
    p(x) = \abs{2 \pi P}^{-\frac{1}{2}} \exp \left(-\frac{1}{2} \T{(x - \bar{x})} P^{-1} (x-\bar{x}) \right)
\end{equation*}
Mit $\bar{x}$: Erwartungswert, $P$: Kovarianzmatrix.

\section{Definitionen}
Ein (lineares) Systemmodell sei gegeben durch:
\begin{equation*}
    x(k+1) = F(k) x(k) + G(k) u(k) + v(k)
\end{equation*}
mit:
\begin{itemize}
    \item $x(k) \in \mathbb{R}^{n \times 1}$: Zustandsvektor
    \item $u(k) \in \mathbb{R}^{j \times 1}$: Steuervektor
    \item $u(k) \in \mathbb{R}^{j \times 1}$: Sequenz von gaußverteiltem, mittelwertfreien weißem Rauschen
    \item $F(k) \in \mathbb{R}^{n \times n}$: Systemmatrix
    \item $G(k) \in \mathbb{R}^{n \times j}$: Steuermatrix
    \item $k \in \mathbb{N}$ Zeitindex der diskreten Zeit
\end{itemize}
Das Modellrauschen hat die Kovarianz:
\begin{equation*}
    Q(k) = \E{v(k)\T{v(k)}}
\end{equation*}

Sowie die Messgleichung:
\begin{equation*}
    z(k) = H(k) x(k) + w(k)
\end{equation*}
mit:
\begin{itemize}
    \item $z(k) \in \mathbb{R}^{m \times 1}$: Messvektor
    \item $H(k) \in \mathbb{R}^{m \times n}$: Messmatrix
    \item $w(k) \in \mathbb{R}^{m \times 1}$: Sequenz von gaußverteiltem, mittelwertfreiem, weißen Rauschen
\end{itemize}
Das Messrauschen hat die Kovarianz:
\begin{equation*}
    R(k) = \E{w(k)\T{w(k)}}
\end{equation*}

Zudem seien die beiden Rauschprozesse (System- und Modellrauschen) statistisch unabhängig, d.h.
die Kreuzkorrelation ergibt sich zu null.

Um abhängigkeiten des Modelrauschens zu modellieren wird das Rauschen z.t. als $\Gamma(k) v(k)$, mit
$\Gamma \in \mathbb{R}^{n \times i}$ gewählt. Die Kovarianz ist dann:
\begin{equation*}
    \E{(\Gamma(k)v(k))\T{(\Gamma(k)v(k))}} = \Gamma(k) Q(k) \T{\Gamma(k)}
\end{equation*}

\section{Kalmanfilter}
\subsection{Prädiktionsschritt}
\begin{itemize}
    \item Zustandsprädiktion:
        \begin{equation*}
            \hat{x}(k+1|k) = F(k) \hat{x}(k|k) + G(k) u(k)
        \end{equation*} 
    \item Kovarianzprädiktion:
        \begin{equation*}
            P(k+1|k) = F(k)P(k|k)\T{F(k)} + Q(k)
        \end{equation*}
    \item Messwertprädiktion:
        \begin{equation*}
            \hat{z}(k+1|k) = H(k+1) \hat{x}(k+1|k)
        \end{equation*}
    \item Kovarianz-Messwertprädiktion:
        \begin{equation*}
            S(k+1) = H(k+1)P(k+1|k)\T{H(k+1)} + R(k+1)
        \end{equation*}
\end{itemize}

\subsection{Innovationsschritt}
\begin{itemize}
    \item Filterverstärkung;
        \begin{equation*}
            K(k+1) = P(k+1|k) \T{H(k+1)} {S(k+1)}^{-1}
        \end{equation*}
    \item Zustandsupdate
        \begin{equation*}
            \hat{x}(k+1|k+1) = \hat{x}(k+1|k) + K(k+1) (z(k+1) - \hat{z}(k+1|k))
        \end{equation*}
    \item Kovarianzupdate
        \begin{equation*}
            P(k+1|k+1) = P(k+1|k) - K(k+1) S(k+1) \T{K(k+1)}
        \end{equation*}
\end{itemize}
