\chapter{Datenassoziation}
\section{Gatting}
Um die Assoziation zu vereinfachen werden im Vorraus unwahrscheinliche Kombinationen ausgeschlossen. 

Für einen Track ist die Region in der eine Messung mit hoher Wahrscheinlichkeit liegt definiert als:
\begin{equation*}
    V(\alpha) = \{z | \T{\gamma}(k+1){S(k+1)}^{-1} \gamma(k+1) < \alpha\}
\end{equation*}
Mit $\alpha$ der Konfidenzschranke (quadratische Mahalanobis-Distanz).

\section{(Local) Nearest Neighbour}
Es wird die Mahalanobis Distanz verwendet um jeder Messung einen Track zuzuordnen, hierbei können doppelungen vorkommen.

\section{Global Nearest Neighbour}
Die Summe der Abstände soll minimiert werden, zudem wird gefordert, dass jede Zuordnung eindeutig ist.

\subsection{Auction Algorithmus}
Zuerst muss für jede Zuordung von Messung und Track die Kosten definiert werden, dies geschiet mithilfe der Mahalanobis Distanz und dem Gate $G$:
\begin{equation*}
    a_{ij} = G - d_{ij}
\end{equation*}
wobei hier $G = \alpha$ angenommen wird.

Zusätzlich wird eine Schranke $\varepsilon$ definiert:
\begin{equation*}
    \varepsilon < \frac{1}{\dim[z]}
\end{equation*}

\subsubsection{Algorithmus}
\begin{enumerate}
    \item Wähle alle Trackpreise zu $P_i=0$
    \item Wähle eine noch nicht zugeordnete Beobachtung $j$, falls keine existiert ist das Ergebnis erreicht
    \item Suche den besten Track $i$ für die Beobachtung ($a_{ij} - P_{i_j} = \max_i \{a_{ij} - P_i\}$)
    \item Falls eine andere Beobachtung zugeordnet war verwerfe diese Zuordnung
    \item Preisupdate: $P_{i_j} = P_{i_j} + y_j + \varepsilon$, mit $y_j$ 
        der Differenz zwischen der besten und der zweitbesten Lösung
    \item Gehe zu Schritt 2
\end{enumerate}
