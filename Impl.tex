\chapter{Aspekte der Implementation}
\section{Numerische Probleme bei der Berechnung von Matrizen}
Kovarianzmatrizen besitzen zwei elementare Eigenschaften:
\begin{itemize}
    \item Symmetrie
    \item Positive Definitheit
\end{itemize}
Durch Rundungsfehler von Maschinenzahlen können diese Eigenschaften verletzt werden, daher müssen
äquivalente Formulierungen gefunden werden bei denen diese Eigenschaften garantiert erhalten bleiben.

Für die Kovarianzprädiktionsgleichung:
\begin{equation*}
    P(k+1|k) = F(k) P(k|k) \T{F(k)} + \Gamma(k) Q(k) \T{\Gamma(k)}
\end{equation*}
treten keine Probleme bezüglich der positiven Definitheit auf, da nur Additionen vorkommen. 
Die Symmetrie kann durch zweckmäßige Implementierung der Matrixmultiplikation garantiert werden.

Bei der Kovarianzinnovationsgleichung:
\begin{equation*}
    P(k+1|k+1) = P(k+1|k) - K(k+1) S(k+1) \T{K(k+1)}
\end{equation*}
kann es zu Problemen bei der positiven Definitheit kommen, durch die äquivalente Formulierung:
\begin{equation*}
    P(k+1|k+1) = \left(I-K(k+1)H(k+1)\right) P(k+1|k) \T{\left(I-K(k+1)H(k+1)\right)} + K(k+1) R(k+1) \T{K(k+1)}
\end{equation*}
der sog. Joseph-Form kann das verhindert werden. Hier ist nur wieder die Symmetrie der Matrizen zu beachten.

\section{Sequentielle Formulierung des Kalmanfilters}
Falls das Messrauschen der einzelnen Messgrößen unabhängig ist, also
\begin{equation*}
    R(k) = \diag(\sigma_1^2(k), \ldots, \sigma_m^2(k))
\end{equation*}
gilt, dann können die Messungen auch als Einzelmessungen eingebracht werden. Dadurch reduziert sich der Rechenaufwand,
da keine Matrix-Inversionen vonnöten sind. Es gibt weniger Probleme durch Rundungsfehler. 
Zudem können Outlier-Messungen ignoriert werden.

Die Messung
\begin{equation*}
    z(k) =
        \begin{pmatrix}
            z_1(k) \\ \vdots \\ z_m(k)
        \end{pmatrix} = 
        \begin{pmatrix}
            \T{h_1(k)} x(k) + w_1(k) \\
            \vdots \\
            \T{h_m(k)} x(k) + w_m(k)
        \end{pmatrix}
\end{equation*}
wird als Sequenz von skalaren Messungen Interpretiert.

Der a-priori Schätzwert (mit $0$ eingebrachten Messungen) wird geschrieben als
\begin{eqnarray*}
    \hat{x}(k|k, 0) &=& \hat{x}(k|k-1) \\
    P(k|k, 0) &=& P(k|k-1)
\end{eqnarray*}

Nun kann der Innovationsschritt durchgeführt werden:
\begin{eqnarray*}
    s(k,i) &=& \T{h_i(k)} P(k|k, i-1) h_i(k) + r_i(k) \\
    k(k,i) &=& \frac{P(k|k, i-1)h_i(k)}{s(k,i)} \\
    \hat{x}(k|k, i) &=& \hat{x}(k|k, i-1) + k(k,i) \left(z_i(k) - \T{h_i(k)} \hat{x}(k|k, i-1)\right) \\
    P(k|k, i) &=& P(k|k, i-1) - k(k, i) \T{h_i(k)} P(k|k, i-1) \\
        &=& P(k|k, i-1) - \frac{P(k|k, i-1)h_i(k) \T{h_i(k)} \T{P(k|k, i-1)}}{\T{h_i(k)} P(k|k,i-1)h_i(k)+r_i(k)}
\end{eqnarray*}
Nach Einbringen aller Messungen gilt:
\begin{eqnarray*}
    \hat{x}(k|k) &=& \hat{x}(k|k, n_m) \\
    P(k|k) &=& P(k|k, n_m)
\end{eqnarray*}

\subsection{Erkennung von Außreisern}
Da alle Messungen einzeln eingebracht werden, ist es möglich einzelne Teile der Messung zu ignorieren wenn sie mehr als
$n \sigma$ ($n=2 \stackrel{\wedge}{=} 95\%$, $n=3 \stackrel{\wedge}{=} 99.73\%$) von der Messwertprädiktion abweichen.

\section{Maßnahmen bei Verletzung der statistischen Voraussetzungen}
\subsection{Autokorreliertes Prozessrauschen - Shaping Filter}
Das System wird durch ein Shaping-Filter erweitert, das aus einer (virtuellen) weißen Rauschsequenz die wirklich
vorhandene farbige Rauschsequenz erzeugt. Dadurch wird die Systemgleichung i.A. komplexer und die Dimension des
Zustandsraums vergrößert.

\subsection{Korreliertes Prozess- und Messrauschen}
Die Systemgleichung wird erweitert durch (die Matrizen müssen nicht zeitinvariant sein, wird hier aber zur Vereinfachung
angenommen):
\begin{eqnarray*}
    x(k+1) &=& Fx(k) + v(k) + T \left( z(k) - H x(k) - w(k)\right) \\
        &=& (F-TH) x(k) + v(k) - T w(k) + T z(k)
\end{eqnarray*}
Durch umdefinition lässt sich die Gleichung schreiben als:
\begin{equation*}
    x(k+1) = F^* x(k) + v^*(k) + u^*(k)
\end{equation*}
mit
\begin{eqnarray*}
    F^* &=& F - TH \\
    v^*(k) &=& v(k) - Tw(k) \\
    u^*(k) &=& Tz(k)
\end{eqnarray*}

Aus der Bedingung, dass die Kreuzkorrelation $0$ sein muss folgt dann:
\begin{equation*}
    T = U R^{-1}
\end{equation*}

\subsection{Autokorreliertes Messrauschen}
Das Messrauschen wird modelliert durch:
\begin{equation*}
    w_c(k+1) = F_c w_c(k) + w_w(k)
\end{equation*}

Anstatt der normalen Messgleichung wird nun eine Differenzmessung durchgeführt:
\begin{equation*}
    y(k) = z(k+1) - F_c z(k) = HFx(k) + Hv(k) +w_w(k) -F_c H x(k)
\end{equation*}
durch umdefinieren lässt sich die Messgleichung dann schreiben als:
\begin{equation*}
    y(k) = H^* x(k) + w(k)
\end{equation*}
mit
\begin{eqnarray*}
    H^* &=& HF - F_c H \\
    w(k) &=& H v(k) + w_w(k) 
\end{eqnarray*}
Nun ist zwar das Messrauschen weißes Rauschen, dafür sind aber Mess- und Prozessrauschen korreliert, d.h. im nächsten
Schritt muss das Vorgehen aus dem vorherigen Abschnitt angewandt werden.
